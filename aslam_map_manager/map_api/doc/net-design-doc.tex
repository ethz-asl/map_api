\documentclass{article}
\usepackage[utf8]{inputenc}
\usepackage{todonotes}
\usepackage{a4wide}

\title{Network design doc for review by Christian}
\author{Titus Cieslewski}
\date{Spring 2014}

\setlength{\parindent}{0mm}
%\setlength{\parskip}{3mm}

\begin{document}

\maketitle

\section{Overview}

The Map API is to provide a mean for robots to collaborate on maps on a large
scale, in a distributed (peer to peer) manner. The reference mapping algorithm
is (collaborative) keyframe bundle adjustment, but the design should allow the 
use of any algorithm.

In order to achieve that, the data that is shared among the robots is structured
into tables that can be specified by the algorithms.

In other words, thus, the goal is to develop an application-specific peer to
peer database system.

\subsection{Keyframe bundle adjustment}

To illustrate use cases in this document, I will use a hypothetical application
performing collaborative keyframe bundle adjustment. 

The main data unit of the algorithm is what we call a mission. A mission
represents a single robot's trajectory in the world, along with its perception
during that trajectory.

The first thing that is known about a mission is the raw sensor data. We assume
that the robots participating in the application dispose of at least a camera
and an inertial measurement unit (IMU). Without going into too much detail, the
algotithm then generates a first 3D estimation of the trajectory and of salient
features in the environment (landmarks). The trajectory is represented by a
graph where each vertex represents a reasonably sampled pose (keyframe) of the
robot during its trajectory, and each edge a spatial transformation between two 
keyframes. Such a graph is also called "pose graph".

Once a first estimate is given, the pose graph and landmark...


\end{document}
